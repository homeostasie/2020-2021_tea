\documentclass[12pt]{article}
\usepackage{geometry} % Pour passer au format A4
\geometry{hmargin=1cm, vmargin=1cm} % 

% Page et encodage
\usepackage[T1]{fontenc} % Use 8-bit encoding that has 256 glyphs
\usepackage[english,french]{babel} % Français et anglais
\usepackage[utf8]{inputenc} 

\usepackage{lmodern}
\setlength\parindent{0pt}

% Graphiques
\usepackage{graphicx,float,grffile}

% Maths et divers
\usepackage{amsmath,amsfonts,amssymb,amsthm,verbatim}
\usepackage{multicol,enumitem,url,eurosym,gensymb,multido}

% Sections
\usepackage{sectsty} % Allows customizing section commands
\allsectionsfont{\centering \normalfont\scshape}

% Tête et pied de page

\usepackage{fancyhdr} 
\pagestyle{fancyplain} 

\fancyhead{} % No page header
\fancyfoot{}

\\renewcommand{\headrulewidth}{0pt} % Remove header underlines
\\renewcommand{\footrulewidth}{0pt} % Remove footer underlines

\newcommand{\horrule}[1]{\rule{\linewidth}{#1}} % Create horizontal rule command with 1 argument of height

\newcommand{\Pointilles}[1][3]{%
  \multido{}{#1}{\makebox[\linewidth]{\dotfill}\\[\parskip]
}}

\setlength{\columnseprule}{1pt}

\begin{document}

\textbf{Nom, Prénom :} \hspace{8cm} \textbf{Classe :} \hspace{3cm} \textbf{Date :}\\

\begin{center}
  \textit{L'imagination est une grande puissance dont généralement on ne tient pas assez compte dans la société.}  - \textbf{Mikhaïl Bakounine}
\end{center}

\subsection*{Ex1 - Multiplications de fractions}

\textit{\textbf{Calculer} - Vous ne devez pas simplifier}

\begin{multicols}{4}
  \begin{enumerate}
    \item[1a.] A = $\dfrac{24}{7} \times \dfrac{3}{36}$
    \item[1b.] B = $\dfrac{21}{64} \times \dfrac{40}{63}$
    \item[1c.] C = $\dfrac{54}{35} \times \dfrac{5}{18}$
    \item[1d.] D = $\dfrac{5}{14} \times \dfrac{4}{35}$
  \end{enumerate} 
\end{multicols}
  

\subsection*{Ex2 - Critères de divisibilité}

\textit{\textbf{Cours} - réécrire et compléter les quatre phrases.}

\begin{multicols}{2}
  \begin{enumerate}
    \item[2a.] Un nombre est divisible par  2 si \dots
    \item[2b.] Un nombre est divisible par  5 si \dots
    \item[2c.] Un nombre est divisible par 10 si \dots
    \item[2d.] Un nombre est divisible par  3 si \dots
  \end{enumerate} 
\end{multicols}

\subsection*{Ex3 - Critères de divisibilité}

\textit{\textbf{A} - Cocher les bonnes réponses.}

\\renewcommand{\arraystretch}{1}

\begin{tabular}{c@{ est divisible : \kern1cm}l@{ par 2\kern1cm}l@{ par 3\kern1cm}l@{ par 5\kern1cm}l@{ par 10}}
  414 & $\square$ & $\square$ & $\square$ & $\square$ \\
  192 & $\square$ & $\square$ & $\square$ & $\square$ \\
  740 & $\square$ & $\square$ & $\square$ & $\square$ \\
  283 & $\square$ & $\square$ & $\square$ & $\square$ \\
  425 & $\square$ & $\square$ & $\square$ & $\square$ \\
\end{tabular}

\textit{\textbf{B} - Cocher les bonnes réponses.}

\\renewcommand{\arraystretch}{2}

\begin{tabular}{c@{ est divisible : \kern1cm}l@{ par 2\kern1cm}l@{ par 3\kern1cm}l@{ par 5\kern1cm}l@{ par 10}}
  $\dfrac{35}{250}$ & $\square$ & $\square$ & $\square$ & $\square$ \\
  $\dfrac{123}{234}$ & $\square$ & $\square$ & $\square$ & $\square$ \\
  $\dfrac{458}{729}$ & $\square$ & $\square$ & $\square$ & $\square$ \\
  $\dfrac{132}{144}$ & $\square$ & $\square$ & $\square$ & $\square$ \\
\end{tabular}


\subsection*{Ex4 - Simplifier les fractions}

\textit{\textbf{Il faut rédiger les étapes}}

\begin{multicols}{3}
  \begin{enumerate}
    \item[4a.] A = $\dfrac{780}{420}$
    \item[4b.] B = $\dfrac{104}{56}$
    \item[4c.] C = $\dfrac{1170}{630}$
  \end{enumerate} 
\end{multicols}


\subsection*{Ex5 - Additionner et soustraire les fractions}

\textit{\textbf{Il faut rédiger les étapes}}

\begin{multicols}{4}
  \begin{enumerate}
    \item[5a.]  = $\dfrac{4}{15} + \dfrac{3}{5}$
    \item[5b.]  = $\dfrac{10}{3} + 10$
    \item[5c.]  = $\dfrac{4}{45} + \dfrac{6}{9}$
    \item[5d.]  = $\dfrac{3}{6} + \dfrac{6}{42}$
    \item[5e.]  = $8 - \dfrac{3}{6}$
    \item[5f.]  = $\dfrac{63}{14} - \dfrac{2}{7}$
    \item[5g.]  = $\dfrac{1}{16} + 1$
    \item[5h.]  = $1 - \dfrac{3}{7}$
    \end{enumerate} 
\end{multicols}

\newpage


\textbf{Nom, Prénom :} \hspace{8cm} \textbf{Classe :} \hspace{3cm} \textbf{Date :}\\

\begin{center}
  \textit{L'imagination est une grande puissance dont généralement on ne tient pas assez compte dans la société.}  - \textbf{Mikhaïl Bakounine}
\end{center}

\subsection*{Ex1 - Multiplications de fractions}

\textit{\textbf{Calculer} - Vous ne devez pas simplifier}

\begin{multicols}{4}
  \begin{enumerate}
    \item[1a.] A = $\dfrac{21}{6} \times \dfrac{8}{37}$
    \item[1b.] B = $\dfrac{22}{65} \times \dfrac{41}{64}$
    \item[1c.] C = $\dfrac{55}{36} \times \dfrac{6}{19}$
    \item[1d.] D = $\dfrac{6}{15} \times \dfrac{5}{36}$
  \end{enumerate} 
\end{multicols}
  

\subsection*{Ex2 - Critères de divisibilité}

\textit{\textbf{Cours} - réécrire et compléter les quatre phrases.}

\begin{multicols}{2}
  \begin{enumerate}
    \item[2a.] Un nombre est divisible par  2 si \dots
    \item[2b.] Un nombre est divisible par  5 si \dots
    \item[2c.] Un nombre est divisible par 10 si \dots
    \item[2d.] Un nombre est divisible par  3 si \dots
  \end{enumerate} 
\end{multicols}

\subsection*{Ex3 - Critères de divisibilité}

\textit{\textbf{A} - Cocher les bonnes réponses.}

\\renewcommand{\arraystretch}{1}

\begin{tabular}{c@{ est divisible : \kern1cm}l@{ par 2\kern1cm}l@{ par 3\kern1cm}l@{ par 5\kern1cm}l@{ par 10}}
  312 & $\square$ & $\square$ & $\square$ & $\square$ \\
  193 & $\square$ & $\square$ & $\square$ & $\square$ \\
  640 & $\square$ & $\square$ & $\square$ & $\square$ \\
  182 & $\square$ & $\square$ & $\square$ & $\square$ \\
  325 & $\square$ & $\square$ & $\square$ & $\square$ \\
\end{tabular}

\textit{\textbf{B} - Cocher les bonnes réponses.}

\\renewcommand{\arraystretch}{2}

\begin{tabular}{c@{ est divisible : \kern1cm}l@{ par 2\kern1cm}l@{ par 3\kern1cm}l@{ par 5\kern1cm}l@{ par 10}}
  $\dfrac{234}{345}$ & $\square$ & $\square$ & $\square$ & $\square$ \\
  $\dfrac{120}{235}$ & $\square$ & $\square$ & $\square$ & $\square$ \\
  $\dfrac{158}{722}$ & $\square$ & $\square$ & $\square$ & $\square$ \\
  $\dfrac{138}{144}$ & $\square$ & $\square$ & $\square$ & $\square$ \\
\end{tabular}


\subsection*{Ex4 - Simplifier les fractions}

\textit{\textbf{Il faut rédiger les étapes}}

\begin{multicols}{3}
  \begin{enumerate}
    \item[4a.] A = $\dfrac{136}{56}$
    \item[4b.] B = $\dfrac{1530}{630}$
    \item[4c.] C = $\dfrac{1020}{420}$
  \end{enumerate} 
\end{multicols}


\subsection*{Ex5 - Additionner et soustraire les fractions}

\textit{\textbf{Il faut rédiger les étapes}}

\begin{multicols}{4}
  \begin{enumerate}
    \item[5a.]  = $\dfrac{4}{6} + \dfrac{3}{12}$
    \item[5b.]  = $\dfrac{8}{5} + 8$
    \item[5c.]  = $\dfrac{3}{40} + \dfrac{7}{8}$
    \item[5d.]  = $\dfrac{4}{48} + \dfrac{11}{6}$
    \item[5e.]  = $9 - \dfrac{3}{9}$
    \item[5f.]  = $\dfrac{73}{21} - \dfrac{3}{7}$
    \item[5g.]  = $\dfrac{1}{15} + 1$
    \item[5h.]  = $1 - \dfrac{3}{8}$
    \end{enumerate} 
\end{multicols}
\end{document}