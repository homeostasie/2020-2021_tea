\documentclass[12pt]{article}
\usepackage{geometry} % Pour passer au format A4
\geometry{hmargin=1cm, vmargin=1cm} % 

% Page et encodage
\usepackage[T1]{fontenc} % Use 8-bit encoding that has 256 glyphs
\usepackage[english,french]{babel} % Français et anglais
\usepackage[utf8]{inputenc} 

\usepackage{lmodern}
\setlength\parindent{0pt}

% Graphiques
\usepackage{graphicx,float,grffile}

% Maths et divers
\usepackage{amsmath,amsfonts,amssymb,amsthm,verbatim}
\usepackage{multicol,enumitem,url,eurosym,gensymb,multido}

% Sections
\usepackage{sectsty} % Allows customizing section commands
\allsectionsfont{\centering \normalfont\scshape}

% Tête et pied de page

\usepackage{fancyhdr} 
\pagestyle{fancyplain} 

\fancyhead{} % No page header
\fancyfoot{}

\renewcommand{\headrulewidth}{0pt} % Remove header underlines
\renewcommand{\footrulewidth}{0pt} % Remove footer underlines

\newcommand{\horrule}[1]{\rule{\linewidth}{#1}} % Create horizontal rule command with 1 argument of height

\newcommand{\Pointilles}[1][3]{%
  \multido{}{#1}{\makebox[\linewidth]{\dotfill}\\[\parskip]
}}

\setlength{\columnseprule}{1pt}

\begin{document}

\textbf{Nom, Prénom :} \hspace{8cm} \textbf{Classe :} \hspace{3cm} \textbf{Date :}\\

\begin{center}
  \textit{En mathématiques, on ne comprend pas les choses, on s'y habitue.}  - \textbf{John Von Neumann}
\end{center}

\subsection*{Ex1 - Leçon}

\textit{\textbf{Donner la définition d'un tableau de proportionnalité.}}

\Pointilles[4]

\subsection*{Ex2 - Démonstration}

\textit{\textbf{Les tableaux sont-ils proportionnels ?}}

\begin{multicols}{3}

  \begin{center}
      \begin{tabular}{|c|c|c|}
        \hline
        42 & 100.8 & 75.6 \\  \hline
        2.5 & 6 & 4.5 \\  \hline
      \end{tabular}
    \end{center}

    \Pointilles[6]

    \begin{center}
      \begin{tabular}{|c|c|c|}
        \hline
        10 & 20 & 30   \\  \hline
        20 & 30 & 40   \\  \hline
      \end{tabular}
    \end{center}

    \Pointilles[6]

    \begin{center}
      \begin{tabular}{|c|c|c|}
        \hline
        2.94 & 58.8 & 30   \\  \hline
           5 &  100 & 51   \\  \hline
      \end{tabular}
    \end{center}    

    \Pointilles[6]

  \end{multicols}

\subsection*{Ex3 - Calculer}

\textit{Les tableaux sont proportionnels.} \textit{\textbf{Calculer et écrire le calcul.}}

\begin{multicols}{4}

\begin{center}
    \begin{tabular}{|c|c|}
      \hline
      1 & 12  \\  \hline
      24 & \phantom{$\dfrac{\dfrac{123456789}{1}}{1}$}  \\  \hline
    \end{tabular}

    \vspace{0.5cm}
    \textbf{Calcul :}\dotfill 
  \end{center}

  \begin{center}
    \begin{tabular}{|c|c|c|}
      \hline
      13 & \phantom{$\dfrac{\dfrac{123456789}{1}}{1}$} \\  \hline
      25 & 13 \\  \hline
    \end{tabular}

    \vspace{0.5cm}
    \textbf{Calcul :}\dotfill 
  \end{center}

  \begin{center}
    \begin{tabular}{|c|c|c|}
      \hline
      124 & 290 \\  \hline
      \phantom{$\dfrac{\dfrac{123456789}{1}}{1}$} & 48 \\  \hline
    \end{tabular}
    
    \vspace{0.5cm}
    \textbf{Calcul :}\dotfill 
  \end{center}

  \begin{center}
    \begin{tabular}{|c|c|c|}
      \hline
      \phantom{$\dfrac{\dfrac{123456789}{1}}{1}$} & 0.12 \\  \hline
      3 & 15 \\  \hline
    \end{tabular}
    
    \vspace{0.5cm}
    \textbf{Calcul :}\dotfill 
  \end{center}

\end{multicols}

\subsection*{Ex4 - Calculer}

\textit{Les tableaux sont proportionnels.} \textit{\textbf{Calculer.}}

\begin{center}
  \begin{tabular}{|c|c|c|c|c|c|c|c|}
    \hline
    2  &               5 &              12 & \phantom{$\dfrac{\dfrac{123456789}{1}}{1}$} &             1.5 & \phantom{$\dfrac{\dfrac{123456789}{1}}{1}$} & \phantom{$\dfrac{\dfrac{123456789}{1}}{1}$} & \phantom{$\dfrac{\dfrac{123456789}{1}}{1}$}  \\  \hline
    25 & \phantom{$\dfrac{\dfrac{123456789}{1}}{1}$} & \phantom{$\dfrac{\dfrac{123456789}{1}}{1}$} &              75 & \phantom{$\dfrac{\dfrac{123456789}{1}}{1}$} &             110 &            2500 &              0.5 \\  \hline
  \end{tabular}
\end{center} 

\begin{center}
  \begin{tabular}{|c|c|c|c|c|c|c|c|}
    \hline
    \phantom{$\dfrac{\dfrac{123456789}{1}}{1}$} &               11 &            0.42 & \phantom{$\dfrac{\dfrac{123456789}{1}}{1}$} &             712 & 3.6 & \phantom{$\dfrac{\dfrac{123456789}{1}}{1}$} & \phantom{$\dfrac{\dfrac{123456789}{1}}{1}$} \\  \hline
    24              &  \phantom{$\dfrac{\dfrac{123456789}{1}}{1}$} & \phantom{$\dfrac{\dfrac{123456789}{1}}{1}$} &             512 & \phantom{$\dfrac{\dfrac{123456789}{1}}{1}$} & 4.8 &            11.8 &             444 \\  \hline
  \end{tabular}
\end{center} 


\newpage


\textbf{Nom, Prénom :} \hspace{8cm} \textbf{Classe :} \hspace{3cm} \textbf{Date :}\\

\begin{center}
  \textit{En mathématiques, on ne comprend pas les choses, on s'y habitue.}  - \textbf{John Von Neumann}
\end{center}

\subsection*{Ex1 - Leçon}

\textit{\textbf{Donner la définition d'un tableau de proportionnalité.}}

\Pointilles[4]

\subsection*{Ex2 - Démonstration}

\textit{\textbf{Les tableaux sont-ils proportionnels ?}}

\begin{multicols}{3}

  \begin{center}
      \begin{tabular}{|c|c|c|}
        \hline
        10 & 20 & 30 \\  \hline
        50 & 60 & 70 \\  \hline
      \end{tabular}
    \end{center}

    \Pointilles[6]

    \begin{center}
      \begin{tabular}{|c|c|c|}
        \hline
        42 & 67.2 & 100.8 \\  \hline
         5 &    8 &    12 \\  \hline
      \end{tabular}
    \end{center}

    \Pointilles[6]

    \begin{center}
      \begin{tabular}{|c|c|c|}
        \hline
        2.2 & 13.7 & 27 \\  \hline
          4 &   25 & 49 \\  \hline
      \end{tabular}
    \end{center}    

    \Pointilles[6]

  \end{multicols}

\subsection*{Ex3 - Calculer}

\textit{Les tableaux sont proportionnels.} \textit{\textbf{Calculer et écrire le calcul.}}

\begin{multicols}{4}

\begin{center}
    \begin{tabular}{|c|c|}
      \hline
      1 & 14  \\  \hline
      22 & \phantom{$\dfrac{\dfrac{123456789}{1}}{1}$}  \\  \hline
    \end{tabular}

    \vspace{0.5cm}
    \textbf{Calcul :}\dotfill 
  \end{center}

  \begin{center}
    \begin{tabular}{|c|c|c|}
      \hline
      23 & \phantom{$\dfrac{\dfrac{123456789}{1}}{1}$} \\  \hline
      35 & 23 \\  \hline
    \end{tabular}

    \vspace{0.5cm}
    \textbf{Calcul :}\dotfill 
  \end{center}

  \begin{center}
    \begin{tabular}{|c|c|c|}
      \hline
      224 & 190 \\  \hline
      \phantom{$\dfrac{\dfrac{123456789}{1}}{1}$} & 45 \\  \hline
    \end{tabular}
    
    \vspace{0.5cm}
    \textbf{Calcul :}\dotfill 
  \end{center}

  \begin{center}
    \begin{tabular}{|c|c|c|}
      \hline
      \phantom{$\dfrac{\dfrac{123456789}{1}}{1}$} & 0.15 \\  \hline
      6 & 21 \\  \hline
    \end{tabular}
    
    \vspace{0.5cm}
    \textbf{Calcul :}\dotfill 
  \end{center}

\end{multicols}

\subsection*{Ex4 - Calculer}

\textit{Les tableaux sont proportionnels.} \textit{\textbf{Calculer.}}

\begin{center}
  \begin{tabular}{|c|c|c|c|c|c|c|c|}
    \hline
    2  &               7 &              13 & \phantom{$\dfrac{\dfrac{123456789}{1}}{1}$} &             1.4 & \phantom{$\dfrac{\dfrac{123456789}{1}}{1}$} & \phantom{$\dfrac{\dfrac{123456789}{1}}{1}$} & \phantom{$\dfrac{\dfrac{123456789}{1}}{1}$}  \\  \hline
    21 & \phantom{$\dfrac{\dfrac{123456789}{1}}{1}$} & \phantom{$\dfrac{\dfrac{123456789}{1}}{1}$} &              65 & \phantom{$\dfrac{\dfrac{123456789}{1}}{1}$} &             90 &            3500 &              0.5 \\  \hline
  \end{tabular}
\end{center} 

\begin{center}
  \begin{tabular}{|c|c|c|c|c|c|c|c|}
    \hline
    \phantom{$\dfrac{\dfrac{123456789}{1}}{1}$} &               13 &            0.34 & \phantom{$\dfrac{\dfrac{123456789}{1}}{1}$} &             612 & 3.4 & \phantom{$\dfrac{\dfrac{123456789}{1}}{1}$} & \phantom{$\dfrac{\dfrac{123456789}{1}}{1}$} \\  \hline
    22              &  \phantom{$\dfrac{\dfrac{123456789}{1}}{1}$} & \phantom{$\dfrac{\dfrac{123456789}{1}}{1}$} &             512 & \phantom{$\dfrac{\dfrac{123456789}{1}}{1}$} & 4.5 &            11.9 &             644 \\  \hline
  \end{tabular}
\end{center} 


\end{document}