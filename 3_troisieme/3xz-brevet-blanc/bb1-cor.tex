\documentclass[10pt]{article}
\usepackage[T1]{fontenc}
\usepackage[utf8]{inputenc}%ATTENTION codage UTF8
\usepackage{fourier}
\usepackage[scaled=0.875]{helvet}
\renewcommand{\ttdefault}{lmtt}
\usepackage{amsmath,amssymb,makeidx}
\usepackage[normalem]{ulem}
\usepackage{diagbox,fancybox,tabularx,booktabs,colortbl}
\usepackage{pifont,multirow,dcolumn,enumitem,textcomp,lscape}
\newcommand{\euro}{\eurologo{}}
\usepackage{graphics,graphicx}
\usepackage{pstricks,pst-plot,pst-tree,pstricks-add}
\usepackage[left=1.5cm, right=1.5cm, top=1cm, bottom=1.5cm]{geometry}
\newcommand{\R}{\mathbb{R}}
\newcommand{\N}{\mathbb{N}}
\newcommand{\D}{\mathbb{D}}
\newcommand{\Z}{\mathbb{Z}}
\newcommand{\Q}{\mathbb{Q}}
\newcommand{\C}{\mathbb{C}}
\usepackage{scratch}
\renewcommand{\theenumi}{\textbf{\arabic{enumi}}}
\renewcommand{\labelenumi}{\textbf{\theenumi.}}
\renewcommand{\theenumii}{\textbf{\alph{enumii}}}
\renewcommand{\labelenumii}{\textbf{\theenumii.}}
\newcommand{\vect}[1]{\overrightarrow{\,\mathstrut#1\,}}
\def\Oij{$\left(\text{O}~;~\vect{\imath},~\vect{\jmath}\right)$}
\def\Oijk{$\left(\text{O}~;~\vect{\imath},~\vect{\jmath},~\vect{k}\right)$}
\def\Ouv{$\left(\text{O}~;~\vect{u},~\vect{v}\right)$}
\usepackage{fancyhdr} 
%\fancyfoot[C]{\textbf{page \thepage}} 
\pagestyle{fancyplain} 

\fancyhead{} % No page header
%\fancyfoot{}

\renewcommand{\headrulewidth}{0pt} % Remove header underlines
%\renewcommand{\footrulewidth}{0pt} % Remove footer underlines
\usepackage[french]{babel}
\usepackage[dvips]{hyperref}

\usepackage[np]{numprint}
%\frenchbsetup{StandardLists=true}

\newcommand{\HRule}{\rule{\linewidth}{0.5mm}}


\begin{document}

\setlength{\columnseprule}{1pt}


\subsection*{Exercice 1 \hfill \textit{12 points}}

\begin{enumerate}
    \item[1.] a. 120 euros
    \item[2.] a. 935 Méga-octet
    \item[3.]  c. inférieur.
\end{enumerate}    
\subsection*{Exercice 2 \hfill \textit{14 points}}


\begin{enumerate}
    \item Les points du graphique ne sont pas alignés. Il ne s'agit donc pas d'une situation de proportionnalité.
    \item 
        \begin{enumerate}
            \item La randonnée a duré  $7$ heures.
            \item La famille a parcouru 20~km.
            \item Le point d'abscisse 6 a une ordonnée de 18 : au bout de six heures la famille a parcouru 18~km.
            \item Le point d'ordonnée 8 a pour abscisse 3 : la famille a parcouru 8 km en 3 heures.
            \item Entre la 4\up{e} et la 5\up{e} heure la distance parcourue n'a pas augmenté : ceci signifie que la famille s'est arrêtée.		
        \end{enumerate}
    \item Un randonneur expérimenté parcourt $7 \times 4 = 28$~km en 7 heures. La famille n'en a fait que 20 : elle n'est pas expérimentée.
    \end{enumerate}
    

\subsection*{Exercice 3 \hfill \textit{16 points}}



\begin{enumerate}
    \item On obtient $- 7 \to - 5 \to (- 5)^2 = 25$.
    
    \item Les droites (AB) et (DE) sont parallèles, d'après le théorème de Thalès, on peut écrire :
    
    $\dfrac{\text{CB}}{\text{CE}} = \dfrac{\text{CA}}{\text{CD}}$, soit ici $\dfrac{\text{CB}}{1,5} = \dfrac{3,5}{1}$, d'où $\text{CB} = 3,5 \times 1,5 = 5,25$~(cm).
    
    \item Enlever 15\,\%, c'est multiplier par $1 - \dfrac{15}{100} = 1 - 0,15 = 0,85$.
    
    Le nouveau prix est donc : $22 \times 0,85 = 18,70$~(\euro).

    \item $(2x - 3)(4x + 1) = 8x^2 + 2x  - 12x  - 3 = 8x^2 - 10x  - 3$.
\end{enumerate}
    
\subsection*{Exercice 5 \hfill \textit{14 points}}

\begin{enumerate}
\item Dans le triangle ABC isocèle en A, la hauteur (AH) est aussi la médiane, donc BH = HC $= \dfrac{290}{2} = 145$.

Le théorème de Pythagore appliqué au triangle ACH rectangle en H s'écrit :

$\text{AC}^2 = \text{AH}^2 + \text{HC}^2$, soit $342^2 = \text{AH}^2 + 145^2$.

Donc $\text{AH}^2 = 342^2 - 145^2 = (342 + 145)\times (342 - 145) = 487 \times 197 = \np{95939}$.

Conclusion AH $ = \sqrt{\np{95939}} \approx 309,74$, soit 310~cm au centimètre près.
\item 
On a avec (MN) parallèle à (BC) une situation de Thalès. On peut donc écrire :

$\dfrac{\text{AN}}{\text{AC}} = \dfrac{\text{MN}}{\text{BC}}$ ou $\dfrac{165}{342} = \dfrac{\text{MN}}{290}$. On en déduit en multipliant chaque membre par 290 :

MN $ = 290 \times \dfrac{165}{342} = \dfrac{290 \times 165}{342} = \dfrac{2 \times 145 \times 3 \times 55}{2 \times 3 \times 57} = \dfrac{145 \times 55}{57} \approx 139,9$ soit environ 140~cm au centimètre près.
\item %Montrer que le coût minimal d'un tel portique équipé de balançoires s'élève à 196,98 \euro.
Il faut :

-- pour la poutre principale 1 poutre de 4 m ;

-- pour les pieds 4 poutres de 3,5 m ;

-- pour le maintien 2 barres de 1,5 m, soit :

$12,99 + 4 \times 11,75 + 2 \times 3,89 = 12,99 + 47 + 7,68 = 66,67$~(\euro), plus les fixations et les deux balançoires, soit :

$66,67 + 80 + 50 = 197,67$~(\euro).
Ce n'est pas le coût minimal car, pour les barres de maintien au lieu de prendre 2 barres de 1,5~m à 3,89~\euro{}, on peut en prendre une de 3~m à 6,99~\euro{} et la couper en deux.

Le coût est alors : 

$12,99 + 4 \times 11,75 +  6,99 + 80 + 50 = 196,98$~(euro).
\item Ajouter 20\,\%, c'est multiplier par $1 + \dfrac{20}{100} = 1 + 0,20 = 1,2$.

Le prix de vente sera donc : $196,98 \times 1,2 = 236,376 \approx 236,38$~(\euro).

\end{enumerate}


\subsection*{Exercice 4 \hfill \textit{18 points}}


\begin{enumerate}
    \item 
    \begin{tabularx}{\linewidth}{|c|c|*{5}{>{\centering \arraybackslash}X|}}\hline
        &A						&B	&C	&D	&E	&F\\ \hline
    1	&Nombre de demi-journées&1	&2	&3	&4	&5\\ \hline
    2	& Tarif A				&8 	&16	&24	&32	&40\\ \hline
    3	& Tarif B				&35	&40	&45	&50	&55\\ \hline
    \end{tabularx}

    
    \item f(x) = 8x et g(x) = 30 + 5x.
    
    
    \item C'est la fonction linéaire $f$.
    \item \scalebox{0.5}                                              
    { 
    \psset{xunit=0.8cm,yunit=0.08cm,arrowsize=2pt 4}
        \begin{pspicture}(-1,-10)(15,120)
        \multido{\n=0+1}{16}{\psline[linecolor=orange,linewidth=0.15pt](\n,0)(\n,120)}
        \multido{\n=0+10}{13}{\psline[linecolor=orange,linewidth=0.15pt](0,\n)(15,\n)}
        \psaxes[linewidth=1.25pt,Dy=20]{->}(0,0)(0,0)(15,120)
        \psplot[plotpoints=2000,linewidth=1.25pt,linecolor=blue]{0}{15}{5 x mul 30 add}
        \uput[dr](14,100){\blue $\mathcal{C}_g$}
        \psplot[plotpoints=2000,linewidth=1.25pt,linecolor=red]{0}{15}{8 x mul}
        \uput[ul](14,112){\red $\mathcal{C}_f$}
        \uput[r](0,118){Tarif en \euro}
        \uput[u](13,0){Nombre de demi-journées}
        \psline[linewidth=1.2pt,linestyle=dashed,ArrowInside=->](0,100)(14,100)(14,0)
        \psline[linewidth=1.2pt,linestyle=dashed,ArrowInside=->](10,80)(10,0)
        \psline[linewidth=1.2pt,linestyle=dashed,ArrowInside=->](10,80)(0,80)
    \end{pspicture} }
    \item %Déterminer le nombre de demi-journées d'activités pour lequel le tarif A est égal au tarif B.
    $\bullet~~$\emph{Graphiquement} (ce qui semble demandé) : on voit que pour $x = 10$ le prix à payer est le même avec les deux formules : 80~\euro.
    
    $\bullet~~$\emph{Par la calcul} Il faut résoudre dans $\N$ l'équation : 
    
    $f(x) = g(x)$ ou $8x = 5x + 30$ ou $3x = 30$ et enfin en multipliant chaque membre par $\dfrac{1}{3}$, \: $x = 10$.
    \item %Avec un budget de 100~\euro, déterminer le nombre maximal de demi-journées auxquelles on peut participer.
    
    %Décrire la méthode choisie.
    $\bullet~~$\emph{Graphiquement} 
    
    La droite d'équation $y = 100$ coupe $\mathcal{C}_g$ en un point d'abscisse maximal, soit $x = 14$.
    
    Avec 100~\euro{} il vaut mieux choisir la formule B ; on aura 14 demi-journées.
    
    $\bullet~~$\emph{Par le calcul}
    
    On résout $100 = f(x)$ soit $100 = 8x$ ou $25 = 2x$, soit $x = 12,5$, donc en fait 12 demi-journées.
    
    On résout ensuite $100 = g(x)$ soit $100 = 5x + 30$ soit $70 = 5x$ c'est-à-dire $5 \times 14 = 5 \times x$, donc $14 = x$.
    \end{enumerate}





\subsection*{Exercice 6 \hfill \textit{6 points}}


\begin{enumerate}
\item \begin{scratch}
    \blockinit{quand \greenflag est cliqué}
    \blockmove{s’orienter à \ovalnum{\selectmenu{90}}}
    \blockpen{stylo en position d’écriture}
    \blockinfloop{répéter \ovalnum{3} fois}
    {
    \blockmove{avancer de \ovalnum{80}}
    \blockmove{tourner \turnleft{} de \ovalnum{120} degrés}
    }
    \end{scratch}

\item Il suffit de compter le nombre de segments tracés : 12. Seule la figure 2 convient.
\end{enumerate}

\subsection*{Exercice 7 \hfill \textit{14 points}}

\begin{enumerate}
\item Le niveau d'eau a frôlé les 6~m  vers 8 h et un peu après 20 h.
\item Il y avait 5 m d'eau à 6~h, 10~h 30, 18~h et 23~h.
\item
	\begin{enumerate}
		\item Entre la marée haute et la marée basse, il s'est écoulé 14 h 30 - 8 h 16 = 6 h 14.
		\item La hauteur de la marée (le marnage) a été $5,89 - 0,90 = 4,99$~m.
	\end{enumerate}

\item On a vu que la marée était de 4,99~m, donc le coefficient de marée est égal à :

$C = \dfrac{4,99}{5,34}\times 100 \approx 93$ : c'était donc une marée de vives-eaux.
\end{enumerate}


\end{document}