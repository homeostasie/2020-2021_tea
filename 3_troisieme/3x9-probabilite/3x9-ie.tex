\documentclass[11pt]{article}
\usepackage{geometry} % Pour passer au format A4
\geometry{hmargin=1cm, vmargin=1cm} % 

% Page et encodage
\usepackage[T1]{fontenc} % Use 8-bit encoding that has 256 glyphs
\usepackage[english,french]{babel} % Français et anglais
\usepackage[utf8]{inputenc} 

\usepackage{lmodern}
\setlength\parindent{0pt}

% Graphiques
\usepackage{graphicx,float,grffile}

% Maths et divers
\usepackage{amsmath,amsfonts,amssymb,amsthm,verbatim}
\usepackage{multicol,enumitem,url,eurosym,gensymb}

% Sections
\usepackage{sectsty} % Allows customizing section commands
\allsectionsfont{\centering \normalfont\scshape}

% Tête et pied de page

\usepackage{fancyhdr} 
\pagestyle{fancyplain} 

\fancyhead{} % No page header
\fancyfoot{}

\renewcommand{\headrulewidth}{0pt} % Remove header underlines
\renewcommand{\footrulewidth}{0pt} % Remove footer underlines

\newcommand{\horrule}[1]{\rule{\linewidth}{#1}} % Create horizontal rule command with 1 argument of height

%----------------------------------------------------------------------------------------
%	Début du document
%----------------------------------------------------------------------------------------

\begin{document}

%----------------------------------------------------------------------------------------
% RE-DEFINITION
%----------------------------------------------------------------------------------------
% MATHS
%-----------

\newtheorem{Definition}{Définition}
\newtheorem{Theorem}{Théorème}
\newtheorem{Proposition}{Propriété}

% MATHS
%-----------
\renewcommand{\labelitemi}{$\bullet$}
\renewcommand{\labelitemii}{$\circ$}
%----------------------------------------------------------------------------------------
%	Titre
%----------------------------------------------------------------------------------------

\setlength{\columnseprule}{1pt}

\textbf{Nom, Prénom :} \hspace{8cm} \textbf{Classe :} \hspace{3cm} \textbf{Date :}\\
\vspace{-0.8cm}
\begin{center}
  \textit{Les propositions mathématiques sont reçues comme vraies parce que personne n'a intérêt qu'elles soient fausses.}  - \textbf{Montesquieu}
\end{center}
\vspace{-0.8cm}

\subsection*{Ex1 - Statistique}
Calculer la moyenne, la médiane et l'étendue.
\begin{multicols}{2}
  1. Données : 10 ; 8 ; -6 ; 4 ; 4; -20 ; 34; 10 ; 10 ; 16 ; 2 ; 0
  2.\begin{tabular}{|c|c|c|c|c|c|c|}
      \hline
      Données  & 0 & 1 & 2 & 3 & 4 & 5 \\  \hline
      Effectif & 8  & 12  & 14  & 8  & 34  & 4 \\  \hline
    \end{tabular}
  \end{multicols}

  \horrule{1px}

\begin{multicols}{2}
  \subsection*{Ex2 - Boules}

  Dans une urne, il y a 4 boules jaunes (J), 5 boules bleues (B) et 5 boules rouges (R), indiscernables au toucher. On tire successivement et sans remise deux boules.
  \begin{enumerate}
  \item Quelle est la probabilité de tirer une boule bleue au premier tirage?
  \item Construire un arbre des probabilités décrivant l'expérience aléatoire.
  \item Quelle est la probabilité que la première boule soit rouge et la deuxième soit bleue?
  \item Quelle est la probabilité que la deuxième boule soit jaune ?
  \item Quelle est la probabilité de tirer deux boules de même couleur ?
  \end{enumerate}

  \subsection*{Ex3 - Pièce}

  On lance trois fois une pièce de monnaie et on regarde sur quelle face celle-ci tombe. On note P s’il s’agit de Pile et F s’il s’agit de Face. 
  PPF signifie que la pièce est tombée sur Pile, Pile et Face, dans cet ordre-là.

  \begin{enumerate}
  \item Quelles sont les issues possibles ?
  \item Quelle est la probabilité de faire trois fois pile : PPP ? 
  \item On vient de faire 2 fois pile. Quelle est la probabilité de faire une troisième fois pile ? \textit{(Justifier)}
  \item Quelle est la probabilité de faire au moins une fois sur Pile ? \textit{(Justifier)}
  \end{enumerate}
\end{multicols}

\vspace{-0.4cm}
\horrule{1px}
\vspace{-0.8cm}

\begin{multicols}{2}
  \subsection*{Ex4 - Dé}

  On lance deux dés à six faces, numérotées de 1 à 6, puis on conserve uniquement le plus grand résultat. Par exemple, on obtient 2 et 4. Je conserve 4.

  \begin{enumerate}
  \item Quelles sont les issues possibles ?
  \item Compléter le tableau ci-contre, indiquant le nombre que l’on retient suivant le résultat des deux dés
  \item En déduire la probabilité associée à chaque issue de cette expérience. 
    \begin{center}
      \begin{tabular}{|c|c|c|c|c|c|c|}
        \hline
        & 1 & 2 & 3 & 4 & 5 & 6 \\
        \hline
        1 &   &   &   &   &   &\\  
        \hline
        2 &   &   &   &   &   &\\  
        \hline
        3 &   &   &   &   &   &\\  
        \hline
        4 &   &   &   &   &   &\\  
        \hline
        5 &   &   &   &   &   &\\  
        \hline
        6 &   &   &   &   &   &\\
        \hline     
      \end{tabular}
    \end{center}
  \item Quelle est l'issue la plus probable. \textit{(celle qui a le plus de chance de se réaliser.)}
  \item Quelle est la probabilité de faire 5 ou plus ?  \textit{(Justifier)}
  \end{enumerate}
\end{multicols}

\vspace{-0.4cm}
\horrule{1px}
\vspace{-0.8cm}

\begin{multicols}{2}
  \subsection*{Ex 5 - Bonbons}
  Sam préfère les bonbons bleus. 
  Dans son paquet de 500 bonbons, 140 sont bleus, les autres sont rouges, jaunes ou verts.

  \begin{enumerate}
  \item Quelle est la probabilité qu’il pioche au hasard un bonbon bleu dans son paquet ?
  \item 20\% des bonbons de ce paquet sont rouges. Combien y a-t-il de bonbons rouges ?
  \item Sachant qu’il y a 130 bonbons verts dans ce paquet, Sam a-t-il plus de chance de piocher au hasard un bonbon vert ou un bonbon jaune ?
  \item Aïcha avait acheté le même paquet il y a quinze jours, il ne lui reste que 130 bonbons bleus,100 jaunes, 60 rouges et 100 verts. Elle dit à Sam : \og Tu devrais piocher dans mon paquet plutôt que dans le tien, tu aurais plus de chance d’obtenir un bleu \fg.
    A-t-elle raison? 
  \end{enumerate}
\end{multicols}


\newpage

\textbf{Nom, Prénom :} \hspace{8cm} \textbf{Classe :} \hspace{3cm} \textbf{Date :}\\
\vspace{-0.8cm}
\begin{center}
  \textit{Les propositions mathématiques sont reçues comme vraies parce que personne n'a intérêt qu'elles soient fausses.}  - \textbf{Montesquieu}
\end{center}
\vspace{-0.8cm}

\subsection*{Ex1 - Statistique}
Calculer la moyenne, la médiane et l'étendue.
\begin{multicols}{2}
1. Données : 12 ; 6 ; -8 ; 4 ; 4; 30 ; -34; -10 ; -10 ; 6 ; 16 ; 0 \\
2.    \begin{tabular}{|c|c|c|c|c|c|c|}
      \hline
      Données  & 0 & 1 & 2 & 3 & 4 & 5 \\  \hline
      Effectif & 5  & 21  & 25  & 15 & 45  & 9 \\  \hline
    \end{tabular}
\end{multicols}

\horrule{1px}

\begin{multicols}{2}
  \subsection*{Ex2 - Boules}
  Dans une urne, il y a 3 boules vertes (V), 3 boules rouges (R) et 6 boules jaunes (J), indiscernables au toucher. On tire successivement et sans remise
  deux boules.
  \begin{enumerate}
  \item Quelle est la probabilité de tirer une boule rouge au premier tirage?
  \item Construire un arbre des probabilités décrivant l'expérience aléatoire.
  \item Quelle est la probabilité que la première boule soit jaune et la deuxième soit rouge?
  \item Quelle est la probabilité que la deuxième boule soit verte ?
  \item Quelle est la probabilité de tirer deux boules de même couleur ?
  \end{enumerate}

  \subsection*{Ex3 - Pièce}

  On lance trois fois une pièce de monnaie et on regarde sur quelle face celle-ci tombe. On note P s’il s’agit de Pile et F s’il s’agit de Face. 
  PPF signifie que la pièce est tombée sur Pile, Pile et Face, dans cet ordre-là.

  \begin{enumerate}
  \item Quelles sont les issues possibles ?
  \item Quelle est la probabilité de faire trois fois face : FFF ?  
  \item On vient de faire 2 fois face. Quelle est la probabilité de faire une troisième fois face ? \textit{(Justifier)}
  \item Quelle est la probabilité de faire au moins une fois sur face ? \textit{(Justifier)}
  \end{enumerate}
\end{multicols}

\vspace{-0.4cm}
\horrule{1px}
\vspace{-0.8cm}

\begin{multicols}{2}
  \subsection*{Ex4 - Dé}

  On lance deux dés à six faces, numérotées de 1 à 6, puis on conserve uniquement le plus petit résultat. Par exemple, on obtient 2 et 4. Je conserve 2.

  \begin{enumerate}
  \item Quelles sont les issues possibles ?
  \item Compléter le tableau ci-contre, indiquant le nombre que l’on retient suivant le résultat des deux dés
  \item En déduire la probabilité associée à chaque issue de cette expérience. 
    \begin{center}
      \begin{tabular}{|c|c|c|c|c|c|c|}
        \hline
        & 1 & 2 & 3 & 4 & 5 & 6 \\
        \hline
        1 &   &   &   &   &   &\\  
        \hline
        2 &   &   &   &   &   &\\  
        \hline
        3 &   &   &   &   &   &\\  
        \hline
        4 &   &   &   &   &   &\\  
        \hline
        5 &   &   &   &   &   &\\  
        \hline
        6 &   &   &   &   &   &\\
        \hline     
      \end{tabular}
    \end{center}
  \item Quelle est l'issue la plus probable. \textit{(celle qui a le plus de chance de se réaliser.)}
  \item Quelle est la probabilité de faire 3 ou plus ?  \textit{(Justifier)}
  \end{enumerate}
\end{multicols}

\vspace{-0.4cm}
\horrule{1px}
\vspace{-0.8cm}

\begin{multicols}{2}
  \subsection*{Ex 5 - Bonbons}
  Sam préfère les bonbons bleus. 
  Dans son paquet de 600 bonbons, 180 sont bleus, les autres sont rouges, jaunes ou verts.

  \begin{enumerate}
  \item Quelle est la probabilité qu’il pioche au hasard un bonbon bleu dans son paquet ?
  \item 20\% des bonbons de ce paquet sont rouges. Combien y a-t-il de bonbons rouges ?
  \item Sachant qu’il y a 130 bonbons verts dans ce paquet, Sam a-t-il plus de chance de piocher au hasard un bonbon vert ou un bonbon jaune ?
  \item Aïcha avait acheté le même paquet il y a quinze jours, il ne lui reste que 140 bonbons bleus,100 jaunes, 60 rouges et 100 verts. Elle dit à Sam : \og Tu devrais piocher dans mon paquet plutôt que dans le tien, tu aurais plus de chance d’obtenir un bleu \fg.
    A-t-elle raison? 
  \end{enumerate}
\end{multicols}

\end{document}
